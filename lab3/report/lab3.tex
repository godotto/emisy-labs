\documentclass{article}

\usepackage{geometry}
\usepackage{amsmath}

\title{Using LCD display and LEDs with keyboard inputs in EdSim simulator -- an EMISY laboratory}
\author{Maciej Marcinkiewicz (300371)}
\date{31st May 2021}

\newgeometry{lmargin=3.2cm, rmargin=3.2cm, bmargin=2.5cm}

\begin{document}

\maketitle

\section{Introduction}
\subsection{Brief description}
Laboratory's main purpose was to learn how to use switches and keyboard matrices
with microcontrollers and its peripherals. Once again tasks had to be done
in the EdSim5 simulator.

\subsection{Hardware description}
Microcontroller (Atmel’s AT89C4051) is connected to 5 V source and is equipped with 12 MHz clock and simple resetting circuit.
LCD screen's data bus is connected to P1 port. Display's RS and E pins are
connected to P3.0 and P3.1 pins, respectively. Switch button SW0 from switch
bank is connected to P2.0 pin.

\section{Task 1}
\subsection{Assembly code}
\ttfamily
; define constants: \\
; LCD\_BUS is LCD's data bus\\
; LCD\_RS is LCD's R/S pin\\
; LCD\_E is LCD's E pin\\
; SW is pin connected to switch from the switch bank\\
\\
LCD\_BUS     equ     P1\\
LCD\_RS      equ     P3.0\\
LCD\_E       equ     P3.1\\
SW          equ     P2.0\\
\\
mov     R0, \#20     ; initialization - wait for approx. 30 ms (or even slightly more)\\
\\
initial\_wait:\\
    lcall   long\_delay          ; call function which waits for more than 1.53 ms\\
    djnz    R0, initial\_wait    ; loop that will perform actions 10 times\\
\\
    mov     LCD\_BUS, \#00111000B ; function set\\
    lcall   send\_command\\
    lcall   short\_delay         ; wait for more than 39 us\\
\\
    mov     LCD\_BUS, \#00001110B ; turn display on\\
    lcall   send\_command\\
    lcall   short\_delay         ; wait for more than 39 us\\
\\
    mov     LCD\_BUS, \#00000110B ; entry set mode\\
    lcall   send\_command\\
    lcall   short\_delay         ; wait for more than 39 us\\
\\
not\_pressed:\\
    mov     LCD\_BUS, \#00000001B\\
    lcall   send\_command\\
    lcall   short\_delay         ; wait for more than 39 us\\
    jb      SW, \$               ; if button is not pressed stay here with blank display else go further to display name\\
    \\
    mov     LCD\_BUS, \#'M'       ; write letter "M"\\
    lcall   send\_data\\
    lcall   long\_delay          ; wait for more than 1.53 ms\\
\\
    mov     LCD\_BUS, \#'a'       ; write letter "a"\\
    lcall   send\_data\\
    lcall   long\_delay          ; wait for more than 1.53 ms\\
\\
    mov     LCD\_BUS, \#'c'       ; write letter "c"\\
    lcall   send\_data\\
    lcall   long\_delay          ; wait for more than 1.53 ms\\
\\
    mov     LCD\_BUS, \#'i'       ; write letter "i"\\
    lcall   send\_data\\
    lcall   long\_delay          ; wait for more than 1.53 ms\\
\\
    mov     LCD\_BUS, \#'e'       ; write letter "e"\\
    lcall   send\_data\\
    lcall   long\_delay          ; wait for more than 1.53 ms\\
\\
    mov     LCD\_BUS, \#'j'       ; write letter "j"\\
    lcall   send\_data\\
    lcall   short\_delay         ; wait for more than 39 us\\
\\
    jnb     SW, \$               ; stay here if button is pressed, else go to clear it\\
    jmp     not\_pressed\\
\\
send\_command:       ; send command from LCD's data bus\\
    clr     LCD\_RS  ; set mode to command mode\\
    setb    LCD\_E   ; get data from bus\\
    clr     LCD\_E\\
    ret\\
\\
send\_data:          ; send character data from LCD's data bus\\
    setb    LCD\_RS  ; set mode to write mode\\
    setb    LCD\_E   ; get data from bus\\
    clr     LCD\_E\\
    ret\\
\\
long\_delay:         ; wait for approx. 1.53 ms: (256 us * 2 + 1 us) * 3 + 2 us + 3 us + 1 us\\
    mov     R1, \#3  \\
\\
inner\_loop\_long\_delay:\\
    mov     R2, \#256\\
    djnz    R2, \$                       ; loop that will perform actions 256 times\\
\\
    djnz    R1, inner\_loop\_long\_delay   ; loop that will perform actions 3 times\\
    ret\\
\\
short\_delay:            ; wait for approx. 39 us: 1 us + 18 us * 2 + 2 us\\
    mov     R0, \#18\\
    djnz    R0, \$       ; loop that will perform actions 18 times\\
    ret

\rmfamily

\subsection{Code description}
\emph{Part of description is from the lab 1}\\
Program starts with constants definition. In order to make the code more readable,
pins has been assigned to mnemonic names. Then the main routine starts with display
initialization. Assigning immiediate value to the register and decrementing it makes
a simple loop which is required for delay. In the beginning microcontroller has to wait
for display at least 30 miliseconds (in order to get proper voltage on display).
Program calls \texttt{long\_delay} subroutine 20 times in order to wait this amount of time.

\texttt{long\_delay} decrements number 256 to zero three times. It gives
appoximately 1.53 ms (even more than this). This number is not coincidence -- such delay
will be used later when several characters will be printed on display.

After 30 ms of delay data bus is filled with value responsible for function set. 
Display has to be set to two line mode. Then command is being sent by \texttt{send\_command}
subroutine. It clears RS pin as command is being sent, not data about character. In the end
it sets and then clears E pin. It is a signal for display to get data from data bus. The last
step is a short delay before the next command. \texttt{short\_delay} subroutine
is even simpler than \texttt{long\_delay}. It only decrements to zero value of 18.
It results in delay of more than 39 $\mu$s which is necessary after executing each display's command.

The next command is responsible for turning on the display and showing cursor. It is executed
in the same manner as function set. The last step in initialization process is entering
set mode. It is set to increment mode, so characters are appearing one after the other.

When initialization is finished, program clears the screen and waits in the loop until
the push button is being pressed. When button is pressed, pin is grounded, so program has to
exit the loop when SW bit is cleared.

If program exits the loop, it is writing letters into display's DDRAM (staring from the first postiton of the first line).
Letter's code is written to the data bus and \texttt{send\_data} is called.
It works almost in the same way as \texttt{send\_command}, the only difference is that
it sets R/S pin to turn on writing mode.

After printing all letters, program stays in the another loop. This time it waits for setting
the SW pin which is of course being set after releasing the button. When program exits the loop
it goes back to not\_pressed label when screen is cleared and once again waits for clearing the SW pin.

\section{Task 2}
\subsection{Assembly code}
\ttfamily
; define constants: \\
; MATRIX\_BUS is a port connected to the keyboard matrix's pins\\
; LED is the first LED from the LED bank\\
\\
MATRIX\_BUS  equ     P0\\
LED         equ     P1.0\\
\\
    jmp     init    ; jump to the initialization\\
\\
    ; interrupt handler\\
    org     013h         ; write instruction to 0xB address (interrupt handler)\\
    xrl     LED, \#1     ; use xor on LED pin to change the bit's value to the opposite one\\
    reti                ; return to the infinite loop\\
\\
init:\\
; global interrupt initialization\\
    setb    EA                      ; enable global interrupts\\
    setb    EX1                     ; enable INT1 interrupt\\
    setb    IT1                     ; set INT1 to work with falling edge of the input signal\\
\\
    mov     MATRIX\_BUS, \#01110000B  ; clear all row bits in order to make keyboard working\\
\\
    jmp     \$                       ; infinite loop waiting for interrupt

\rmfamily

\subsection{Code description}
In the beginning macros are being set up and program immiediately jumps to initialization subroutine.
It enables global interrupts and enables INT1 external interrupt, whose pin is connected to the
output of AND gate. Gates inputs are matrix's column pins. Finally, INT1 is set to detect falling edge
and matrix's port is set in a way to clear all row pins. Without the last step keyboard matrix
would not work properly.

Then program waits for an interrupt in the infinite loop. Interrupt happens when
any key is pressed. In the interrupt handler subroutine XOR operation with 1 is operated on
LED pin. With this instruction every handler execution will alter the pin's value.

\section*{Declaration of authorship}
I declare that this piece of work which is the basis for recognition of achieving learning outcomes in the EMISY course was completed on my own.

\end{document}